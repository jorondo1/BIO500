% Options for packages loaded elsewhere
\PassOptionsToPackage{unicode}{hyperref}
\PassOptionsToPackage{hyphens}{url}
%
\documentclass[
]{article}
\usepackage{amsmath,amssymb}
\usepackage{lmodern}
\usepackage{iftex}
\ifPDFTeX
  \usepackage[T1]{fontenc}
  \usepackage[utf8]{inputenc}
  \usepackage{textcomp} % provide euro and other symbols
\else % if luatex or xetex
  \usepackage{unicode-math}
  \defaultfontfeatures{Scale=MatchLowercase}
  \defaultfontfeatures[\rmfamily]{Ligatures=TeX,Scale=1}
\fi
% Use upquote if available, for straight quotes in verbatim environments
\IfFileExists{upquote.sty}{\usepackage{upquote}}{}
\IfFileExists{microtype.sty}{% use microtype if available
  \usepackage[]{microtype}
  \UseMicrotypeSet[protrusion]{basicmath} % disable protrusion for tt fonts
}{}
\makeatletter
\@ifundefined{KOMAClassName}{% if non-KOMA class
  \IfFileExists{parskip.sty}{%
    \usepackage{parskip}
  }{% else
    \setlength{\parindent}{0pt}
    \setlength{\parskip}{6pt plus 2pt minus 1pt}}
}{% if KOMA class
  \KOMAoptions{parskip=half}}
\makeatother
\usepackage{xcolor}
\usepackage[margin=1in]{geometry}
\usepackage{graphicx}
\makeatletter
\def\maxwidth{\ifdim\Gin@nat@width>\linewidth\linewidth\else\Gin@nat@width\fi}
\def\maxheight{\ifdim\Gin@nat@height>\textheight\textheight\else\Gin@nat@height\fi}
\makeatother
% Scale images if necessary, so that they will not overflow the page
% margins by default, and it is still possible to overwrite the defaults
% using explicit options in \includegraphics[width, height, ...]{}
\setkeys{Gin}{width=\maxwidth,height=\maxheight,keepaspectratio}
% Set default figure placement to htbp
\makeatletter
\def\fps@figure{htbp}
\makeatother
\setlength{\emergencystretch}{3em} % prevent overfull lines
\providecommand{\tightlist}{%
  \setlength{\itemsep}{0pt}\setlength{\parskip}{0pt}}
\setcounter{secnumdepth}{-\maxdimen} % remove section numbering
\ifLuaTeX
  \usepackage{selnolig}  % disable illegal ligatures
\fi
\IfFileExists{bookmark.sty}{\usepackage{bookmark}}{\usepackage{hyperref}}
\IfFileExists{xurl.sty}{\usepackage{xurl}}{} % add URL line breaks if available
\urlstyle{same} % disable monospaced font for URLs
\hypersetup{
  pdftitle={Résumé Delmas et al 2019 - BIO500},
  pdfauthor={Jonathan Rondeau-Leclaire, Amélie Harbeck-Bastien, Samuel Forten},
  hidelinks,
  pdfcreator={LaTeX via pandoc}}

\title{Résumé Delmas et al 2019 - BIO500}
\author{Jonathan Rondeau-Leclaire, Amélie Harbeck-Bastien, Samuel
Forten}
\date{2023-03-31}

\begin{document}
\maketitle

\hypertarget{vocabulaire}{%
\subsection{VOCABULAIRE}\label{vocabulaire}}

\begin{itemize}
\tightlist
\item
  G = graph = réseau
\item
  V = sommet (vertex) = Noeud (node) = étudiant
\item
  E = Arc (edge) = collaboration = sous-ensemble de deux éléments de V
\item
  A = matrice d'adjecence (ou d'incidence = pairwise P/A d'interactions
  (ou pondéré ≈ heatmap)

  \begin{itemize}
  \tightlist
  \item
    symétrique avec diag 0 si non-dirigé sans boucle
  \end{itemize}
\end{itemize}

\hypertarget{mesures}{%
\subsection{MESURES}\label{mesures}}

\begin{itemize}
\tightlist
\item
  \(S\) = Ordre = nombre total de noeuds V. Ex. Richesse si espèces
\item
  \(L\) = Taille = nombre total d'arcs
\item
  \(L/S\) = Densité = moyenne d'interactions par noeud

  \begin{itemize}
  \tightlist
  \item
    Attention à l'interprétation, n'est souvent pas uniforme!
  \item
    \(L\) est souvent proportionnel à \(S^2\)
  \end{itemize}
\item
  \(m\) = nombre d'interactions possible dans un réseau (dépend du type
  de réseau!)
\item
  \(C_0\) = Connectance = \(L/m\) (\% interactions réalisées /
  possibles)

  \begin{itemize}
  \tightlist
  \item
    bonne mesure de sensibilité aux perturbations
  \item
    Varie de 0 à 1 selon si A est vide ou pleine
  \end{itemize}
\item
  \(P(k) = N(k)/S\) : probabilité qu'un noeud ait \(k\) arcs; \(N(k)\)
  est le nombre de noeuds ayant \(k\) arcs. (C'est une moyenne en gros!)
\item
  \(k\) = degré = nombre de noeuds voisins
\item
  \(D\) = Diamètre = Le plus long des plus courts chemins entre chaque
  paire de noeuds, où \(d_{i,j}\) représente le nombre d'interactions
  séparant le noeud \(i\) du noeud \(j\). Peut aussi être mesuré en
  distance moyenne entre chaque noeud du réseau pour normaliser par la
  taille du réseau.
\end{itemize}

\hypertarget{motifs}{%
\subsubsection{Motifs:}\label{motifs}}

\begin{itemize}
\item
  \(CC\) = coefficient de regroupement

  \begin{itemize}
  \tightlist
  \item
    \(cc_i = \frac{2N_i}{k_i(k_i-1)}\), où \(N_i\) = nombre
    d'interactions totales entre tous les voisins de \(i\)
  \item
    fraction des interactions réalisées entre les voisins d'un noeud
  \end{itemize}
\item
  Module: Subsystem of non-overlapping and strongly interacting species.
  Pertinence?
\item
  Emboîtement (nestedness): Pas pertinent
\item
  Intervalité: expliquer les interactions d'un réseau par un gradient
  d'un trait commun. Méthodes non présentées
\end{itemize}

\hypertarget{nos-donnuxe9es}{%
\subsection{NOS DONNÉES}\label{nos-donnuxe9es}}

Réseau unipartite simple non-dirigé, possiblement pondéré, Nos données
sont des interactions réalisées Effort d'échantillonage: pratiquement
exhaustif

Aucune boucle (self-edge) Aucun multi-arcs = graph simple

Unipartite, car toutes les paires sont théoriquement possibles. Diffère
des bipartite, où par exemple les arcs sont toujours faits entre des
membres appartenant à deux groupes distincts (ex. proie-pred)

Matrice d'adjecence: \(A_{S,S}\) où \(a_{i,j} = a_{j,i} ∈ {0,1}\)

S = nombre d'étudiants

\hypertarget{ce-quon-pourrait-calculer}{%
\subsection{CE QU'ON POURRAIT
CALCULER}\label{ce-quon-pourrait-calculer}}

\begin{itemize}
\tightlist
\item
  Ordre, Taille, Densité, diamètre
\item
  Connectance; dans un réseau unipartite simple non-dirigé,
  \(m = S*\frac{S-1}{2}\)
\end{itemize}

\hypertarget{ce-quon-pourrait-tester}{%
\subsection{CE QU'ON POURRAIT TESTER}\label{ce-quon-pourrait-tester}}

\begin{itemize}
\tightlist
\item
  Uniformité et normalité de la densité L/S?
\item
  Distribution des degrés est-elle en puissance?
\item
  y a-t-il une intervalité dans le réseau?
\end{itemize}

\end{document}
